\documentclass[12pt,a4paper]{article}
%\usepackage{harvard}
\usepackage{graphicx}
\usepackage{tabularx}
\usepackage{color}
\usepackage{caption}
\usepackage{tabu}
%\usepackage{pgfplots}
%\usepackage{tikz}
%\usepackage{wrapfig,lipsum,booktabs}
%\usepackage{enumitem}

\captionsetup[figure]{labelfont=bf}
\captionsetup[table]{labelfont=bf}

\title{\vspace{-6em}Virtual and Augmented Reality}
\author{pbqk24}
%\date{}


\begin{document}
	
	\maketitle
	
	\section*{Problem 1}
	
	Write a couple of sentences about the functional interfaces created and their inputs/outputs:
	\begin{itemize}
		\item Euler angle $\leftarrow\rightarrow$ quaternion conversions
		\item Quaternion conjugate calculation
		\item Quaternion product calculation
	\end{itemize}
	
	The following functional interfaces were created:
	
	\begin{tabular}{|c|ll|}
		
		\hline
		Functional interface&Inputs&Outputs\\
		Test&test&test\\
		\hline
	\end{tabular}
	
	
	\begin{itemize}
		\setlength{\itemsep}{0em}
		\item Euler angle $\rightarrow$ quaternion conversion
		\begin{itemize}
			\setlength{\itemsep}{0em}
			\item Input:
			\item Output:
		\end{itemize}
		\item Quaternion $\rightarrow$ Euler angle conversion
		\begin{itemize}
			\item Input/Output as above but swapped
		\end{itemize}
		\item Quaternion conjugate calculation
		\begin{itemize}
			\item Input: 
			\item Output:
		\end{itemize}
		\item Quaternion product calculation
		\begin{itemize}
			\item Input:
			\item Output: 
		\end{itemize}
	\end{itemize}
	
	\section*{Problem 3}
	
	Alpha values and their effect on drift compensation for tilt correction.
	
	\section*{Problem 4}
	
	Alpha values and their effect on drift compensation in yaw correction.
	
	\section*{Problem 5}
	
	Include screenshots of orientation tracking results, and comment on the stability of each method.
	
	\section*{Problem 6}
	
	Comment on what you see in the 3D plots of positional tracking results.
	
	%\begin{figure}[h]
	%	\centering
	%	\label{point_operations}
	%	\begin{minipage}{0.45\textwidth}
	%		\centering
	%		\includegraphics[width=0.9\linewidth]{point_addition}
	%	\end{minipage}
	%	\begin{minipage}{0.1\textwidth}
	%		
	%	\end{minipage}
	%	\begin{minipage}{0.45\textwidth}
	%		\centering
	%		\includegraphics[width=0.819\linewidth]{point_doubling}
	%	\end{minipage}
	%	\caption[Figure 1:]{Point addition (left) and doubling (right) in Elliptic Curve Cryptography, \cite{Dams}}
	%\end{figure}
	
	%\begin{wrapfigure}{l}{0.59\linewidth}
	%	\begin{center}
	%		\begin{tikzpicture}[
	%		every node/.style={text=black},
	%		fdesc/.style={anchor=south east,sloped,font=\scriptsize,pos=#1},
	%		fdesc/.default=1,
	%		]
	%		\begin{semilogyaxis}[
	%		width=\linewidth, % scale plot to \linewidth
	%		grid=major, % Display a grid
	%		grid style={dashed,gray!30}, % Set the style
	%		xlabel=Order size (bits), % Set the x-axis label
	%		ylabel=Wall clock time (seconds), % Set the y-axis label
	%		log basis y={10}, % Set the base for logarithm in y axis
	%		log ticks with fixed point
	%		]
	%		\addplot[only marks, color=red] table[x=Value, y=Time, col sep=comma] {rsa_bruteforce.csv}; % RSA plot
	%		\addplot[only marks, color=blue] table[x=order, y=time, col sep=comma] {ecc_bruteforce.csv}; % ECC plot
	%		\addplot [domain= 13:33, blue, dashed] {7.61337*(10^(-7))*(2^(1.01784*x))}node[left]{$y=7.61336*10^{-7}*2^{1.01784x}$}; % ECC trendline
	%		\addplot [domain= 13:40, red, dashed] {1.71328*(10^(-4))*(2^(0.42945*x))}node[right, pos=0]{$y=1.71328*10^{-4}*2^{0.42945x}$}; % %
	%		RSA trendline
	%		\end{semilogyaxis}
	%		\end{tikzpicture}
	%	\end{center}
	%	\caption[Figure 2:]{Average brute-force attack times against varying order/key sizes in elliptic curve cryptography (blue) and RSA (red)}
	%	\label{ECC_BruteForce}
	%\end{wrapfigure}
	
	%\begin{figure}[h!]
	%	\begin{center}
	%		\begin{tikzpicture}
	%		\begin{axis}[
	%		width=\linewidth*0.45, % scale plot to \linewidth
	%		grid=major, % Display a grid
	%		grid style={dashed,gray!30}, % Set the style
	%		xlabel=Number of point addition operations used, % Set the x-axis label
	%		ylabel=Wall clock time (seconds), % Set the y-axis label
	%		]
	%		\addplot[only marks, mark size = 1.5pt, color=blue] table[x=Value, y=Time, col sep=comma] {k_additions.csv};
	%		\addplot[only marks, mark size = 1.5pt, color=red] table[x=Value, y=Time, col sep=comma] {private_additions.csv};
	%		\end{axis}
	%		\end{tikzpicture}%
	%		~%
	%		%
	%		\begin{tikzpicture}
	%		\begin{axis}[
	%		width=\linewidth*0.45, % scale plot to \linewidth
	%		grid=major, % Display a grid
	%		grid style={dashed,gray!30}, % Set the style
	%		xlabel=Number of point doubling operations used, % Set the x-axis label
	%		ylabel=Time (seconds), % Set the y-axis label
	%		]
	%		\addplot[only marks, color=blue] table[x=Value, y=Time, col sep=comma] {k_doublings.csv};
	%		\addplot[only marks, color=red] table[x=Value, y=Time, col sep=comma] {private_doublings.csv};
	%		\end{axis}
	%		\end{tikzpicture}
	%	\end{center}
	%	\caption[Figure 2:]{Time delay of signature generation compared to the number of point addition operations carried out using signature key $k$ (blue) and private key $A_{priv}$ (red)}
	%	\label{Timing_attack_doublings}
	%\end{figure}
	
\end{document}